\documentclass[a4paper,12pt]{article} \usepackage[utf8]{inputenc}
\usepackage[T1]{fontenc} \usepackage[bulgarian]{babel} \usepackage{amsmath,
amssymb} \usepackage{array,colortbl,xcolor} \usepackage{hyperref}
\usepackage{geometry} \usepackage{float} \geometry{margin=1in}

\hypersetup{ pdfborder={0 0 0}, colorlinks=true, urlcolor=blue, linkcolor=blue,
citecolor=blue }

% Use natbib with superscript citations \usepackage[super]{natbib}

\title{WikiSearch} \author{Даниел Халачев } \date

\begin{document}

\maketitle

\section{Задача}

\subsection{Лична мотивация} Да се приложат на практика част от теоретичните
познания, придобити от курсовете по \textit{Проектиране и обработка на естествен
език} и \textit{Изличане на информация}. Запознаване с езика за програмиране
Python чрез изграждане на напълно функционираща система.

\subsection{Техническа задача} \begin{itemize} \item Да се създаде търсачка на
уебсайтове с общо предназначение, която извършва търсенето въз основа на два
възможни подхода: \begin{itemize} \item семантично търсене чрез document
embeddings; \item традиционно индексиране чрез обратен индекс и търсене и
ранкиране чрез TF-IDF и BM25. \end{itemize} \item По възможност да се
имплементират допълнителни функционалности за подобряване на потребителското
изживяване, като се използват изградените индекси: \begin{itemize} \item
предложения за корекции в правописа; \item предложения за допълване на фразата
на търсене. \end{itemize} \item Да се оцени системата чрез обективни метрики.
\end{itemize}

\bigskip % Table for 1.3 and 1.4: the second column cells are greyed out.
\begin{table}[H] \centering \begin{tabular}{|p{0.
45\textwidth}|>{\color{gray}}p{0.45\textwidth}|} \hline \textbf{Проектиране и
обработка на естествен език} & \textbf{Изличане на информация} \\ \hline
Изграждане на модул за премахване на стоп думи, токенизация и лематизация &
Изграждане на web crawler\\ \hline Векторизация на документите & Изграждане на
обратен индекс. \\ \hline Изграждане на векторен индекс & Имплементиране на
търсене и ранкиране на резултатите с BM25\\ \hline Избор на стратегия и
имплементация на търсене по изградения векторен индекс &Предоставяне на
допълване на заявката за търсене и корекция на правописа\\ \hline \end{tabular}
\caption{Конкретни задачи по дисциплините} \end{table}

\section{Данни} За изграждане на системата беше използван корпус от всички
статии в българския поддомейн на Уикипедия (\texttt{bg.wikipedia.org}). Целият
корпус възлиза на 441\,385 статии, от които 302\,500 са конкретни, а останалите
--- пренасочващи. Корпусът беше изтеглен от файла
\texttt{bg-wiki-20250120-pages-articles.xml.bz2}\cite{wiki}

Разархивиран, той има размер от 3.0 GB и съдържа всички страници на \texttt{bg.
wikipedia.org} в WikiMarkup формат и техните метаданни, организирани йерархично
в XML дърво. С цел поетапна многостъпкова обработка, бяха извлечени само данните
на страниците на статии, след което бяха преработени до чист текст, съхранен в
ефективна база данни.

\section{Избрани подходи, техники и експерименти}

\subsection{Лематизация} Премахването на стоп думи, токенизацията и
лематизацията бяха осъществени чрез библиотеката \texttt{spacy} и модела
\texttt{sakelariev/bg-news-lg}\cite{sakelariev_bg_news_lg}.

\subsection{Векторизация} За изграждането на вектори от текстовете на
документите беше използван моделът Alibaba\cite{alibaba_model}, приложен чрез
библиотеката \texttt{sentence-transformers}. Избраният модел върви с
препоръчителния към него токенизатор.

\subsection{Изграждане на индекса} Първоначалният избор за изграждане на
индекса беше популярната библиотека за векторно търсене FAISS. Използването ѝ
породи два проблема: \begin{itemize} \item Липса на поддръжка на множество
стойности (вектори) за един ключ (документ). Проблемът беше решен чрез
поддържане на речник на съответствията между ключа на вектора във FAISS и
документа, от който е получен. \item Лоша производителност --- бавно изграждане
на индекса и бавно търсене в него. За отстраняването на този проблем бяха
изпробвани множество техники: \begin{itemize} \item Максимална ефективност на
индексирането чрез обработка на множество вектори наведнъж; \item Прекратяване
на всички останали процеси на машината, на която се извършва индексирането;
\item Опити с различни видове индекси (HNSW, IndexL2Flat и др.); \item Опит за
извършване на индексирането на мощен сървър --- въпреки това процесът остана
недостатъчно бърз. \end{itemize} \end{itemize} Нито една от приложените техники
не доведе до решение, което може да бъде изпълнено до крайния срок на проекта.

Решението беше замяната на FAISS с USearch\cite{usearch}, библиотека, която за
разлика от FAISS: \begin{itemize} \item Поддържа съпоставянето на множество
вектори с един ключ; \item Предоставя по-добра производителност при изграждането
и търсенето в индекса. \end{itemize} За съжаление, въпреки ускорение на процеса
от 2 до 4 пъти, времето, загубено в опитите с FAISS, не позволи изграждането на
индекс върху всички 441\,385 статии в Уикипедия.

\subsection{Търсене} Размерността на модела за векторизация от 768 не е
достатъчна, за да обезпечи адекватно векторизиране на всички документи, някои от
които са много големи. За целта беше необходимо текстовете да се разделят на
секции, които могат да бъдат векторизирани. Използването на
USearch\cite{usearch} позволи на един документ да бъде съпоставен неограничен
брой вектори. Това обаче породи следния проблем: в резултатите от търсенето един
документ може да се появи многократно (на различни позиции и с различна
релеватност). Бяха разгледани и имплементирани три стратегии за справяне с този
проблем: \begin{itemize} \item \textbf{Max Pooling} --- от всички двойки
(документ, близост) за един документ се избира тази с най-голяма близост; \item
\textbf{Min Pooling} --- от всички двойки (документ, разстояние) за един
документ се избира тази с най-малко разстояние; \item \textbf{Average Pooling}
--- за всяка двойка (документ, близост) се изчислява средната стойност на
близостите. \end{itemize} И трите техники показаха сходни резултати.

\subsection{Оценяване} Първоначалната идея за оценяване на системата върху
целия домейн \texttt{bg.wikipedia.org} вече не беше приложима поради неочаквано
бавното изграждане на индекса. За демонстрация, че индексът работи, бяха избрани
1000 произволни статии, които да бъдат добавени съответно във векторния и
обратния индекс. За златен стандарт беше избрана библиотеката ElasticSearch, с
доказана репутация и ефективност в областта. Беше създаден специален модул за
генериране на заявки за търсене и оценка на резултатите. Могат да бъдат
генерирани групи заявки от два вида с произволен брой и разпределение на всеки
вид в групата: \begin{itemize} \item заглавие на произволен документ \item избор
на най-често срещаните колокации от две думи в произволни документи, които се
срещат поне 3 пъти в текста. \end{itemize}

\section{Резултати} Резултатите от WikiSearch бяха сравнени с резултатите на
ElasticSearch по критериите Precision, Recall и F1. За целта бяха създадени 5
групи от по 30 заявки за търсене. За всяка заявка се изчисляват Precision,
Recall и F1. Чрез тях се изчисляват средните стойности на тези показатели за
всяка група. Оценителят връща окончателен резултат 95\%-доверителен интервал за
всяка метрика Precision, Recall, F1, изчислен върху всички 5 групи от по 30
заявки. По този начин се гарантира представителност на статистическата извадка,
защото са изпълнени условията за прилагането на Централна гранична теорема.

\begin{table}[h] \centering \begin{tabular}{|c|c|c|} \hline Precision & Recall
& F1 \\ \hline 0.82 & 0.76 & 0.79 \\ \hline \end{tabular} \caption{Резултати на
WikiSearch за обратния индекс} \end{table}

\begin{table}[h] \centering \begin{tabular}{|c|c|c|} \hline Precision & Recall
& F1 \\ \hline 0.85 & 0.78 & 0.81 \\ \hline \end{tabular} \caption{Резултати на
WikiSearch за векторния индекс} \end{table}

\section{Бъдеща работа} \begin{itemize} \item Изграждане на индекс върху целия
домейн \texttt{bg.wikipedia.org} и сравнение на резултатите с тези, върнати от
Wikipedia, а не от ElasticSearch\footnote{Това би било съпроводено с други
проблеми в оценяването, защото Wikipedia подобрява резултатите от търсенето чрез
графи на знанието.}. \textit{Забележка: Това би било съпроводено с други
проблеми в оценяването, защото Wikipedia подобрява резултатите от търсенето чрез
графи на знанието.} \item Избор на по-подходящ модел за векторизация на
текстовете, предназначен специално за български език. \item Внедряване на
интелигентно резюмиране на резултатите. \end{itemize} \begin{thebibliography}{9}
\bibitem{wiki} Корпусът е достъпен на адрес \url{https://dumps.wikimedia.
org/bgwiki/20250120/bgwiki-20250120-pages-articles.xml.bz2}.

\bibitem{sakelariev_bg_news_lg} sakelariev/bg-news-lg, \textit{HuggingFace},
\url{https://huggingface.co/sakelariev/bg-news-lg} (accessed: 2025-02-07).

\bibitem{alibaba_model} Alibaba-NLP/gte-multilingual-base, \textit{HuggingFace},
\url{https://huggingface.co/Alibaba-NLP/gte-multilingual-base} (accessed:
2025-02-07).

\bibitem{usearch} USearch, \textit{Github}, \url{https://github.
com/unum-cloud/usearch} (accessed: 2025-02-07). \end{thebibliography}
\end{document}